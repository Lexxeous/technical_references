% LaTeX Coding Reference

\begin{figure}
 \centering
 \label{fig:bar}
 \captionof{figure}{caption_title}
 \includegraphics[width=\linewidth, height=6cm]{image_title.ext}
\end{figure}



\begin{table}
\caption{caption_title}
\centering
\resizebox{0.3\textwidth}{!}{%
	\begin{tabular}{lll} % left(l), right(r), or center(c) aligned for each column
	\toprule
    Col1 & Col2 & Col3\\
    \midrule
    cell & cell & cell\\
    cell & cell & cell\\
    \bottomrule
	\end{tabular}
}
\end{table}


\begin{itemize}
\item \textbf{Item}
	\begin{itemize}
	\item Subitem
	\end{itemize}
\item \textbf{Item}
	\begin{itemize}
  \item Subitem
	\item Subitem
	\end{itemize}
\end{itemize}


% SYNTAX
$something$ % removes spaces and/or does math mode

% COMMANDS
\command % uses the default version of the command (also defined potentially by packages used)
\command* % uses the alternative version of the command

% A
\approx % outputs ≈
\author{}
% B
\begin{media_type}
\bibliographystyle{citation_style} % sets the styling for the biliography ; apacite, mlacite
\bibliography{bib_file} % creates the biblography referenes automatically ; dont use .bib extention
% C
\caption
\captionof{media_type}{}
\centering
\cfoot
\cite{bib_code};
% D
\date % displays date
% E
\end{media_type}
% F
% G
% H
% I
\includegraphics[width= ,height=, scale=, valign=, halign=]{image_title.ext}
\item
% J
% K
% L
\lfoot{}
% M
% N
% O
% P
\paragraph{paragraph_title} % small bold section title
% Q
% R
\resizebox{}
\rfoot
% S
\section
\subsection
\subsubsection
% T
\textsinglequote % outputs '
\textbf % boldface
\textit % italicized
\textwidth
\title{}
\today % outputs todays date
% U
% V
% W
% X
% Y
% Z





% –––––––––––––––––––––––––––––––––––––––––––––––––––––––––––––––––––––––––––––––––––––––––




Coding Reference:

Symbol Reference Webpage:
https://oeis.org/wiki/List_of_LaTeX_mathematical_symbols

Most, if not all, commands in LaTeX start with '\'.

Syntax:
%: comments
$text$: italicize your "text"
X_x: make a subscript
X^x: make a superscript
*{}: cancel enumeration
&=: aligns equal signs (mathmode)
\\: end line

Special Characters:
\cdot: dot product symbol
\dots: elipsis (dot dot dot) (...)
\mu: mu
\sigma: sigma
\infty: infinity (lemniscate)


Math Symbols:
\frac: create a fraction
\sum_{from}^{to}: capital sigma (additive result)
\sqrt: square root
\mathcal: typeset characters



Commands:
\documentclass[]{}: type of paper that you are writing
\usepackage[]{}: specify which libraries to include
\title{paperTitle}: the title of your paper
\author{authorName}: the author of the paper
\begin{area}: begin an area
\end{area}: end an area
\maketitle: include title, author, and date
\section{secTitle}: begins a new section (automatically numbered) [1..n]
\subsection{subSecTitle}: begins a new subsection (automatically numbered)[m.1 .. m.n]
\centering: will center until the end of the figure
\ref{fig.figName}: will add a clickable link to correct figure
\includegraphics[options]{file.extention}
	Options:
		width = :specify the width of the picture
\caption{caption}
\label{fig:figName}	
\textwidth: can specify the width of your inserted picture
\todo[options]{comment}
	Options:
		inline: makes the comment "inline"
		color = :specify the color if the comment
\footnote{footNoteTxt}: creates a footnote
\item: denote an item in a enumerated or bulleted or etc... list
\verb|text|: plain text font (good for fileName.text inserts)
\textit{sentence}: italicise sentence
\textbf{sentence}: bold sentence
\href{url}{shownText}: creates a highlighted link to "url" but shown as "shownText"
\url{url}: creates a highlighted link shown as "url"
\cite{refUsrName}: creates a link to a specific citation
\text{text}: plain text (mathmode)
\paragraph: title a new paragraph
\big{chars}: makes chars bigger
\times: cross multiple symbol
\pagebreak[num]
	-requests a pagebreak with a [0..4] magnitude of demand variable "num"
\newpage
	-ends the curent page



Beginning and Ending Keyword Sections:
document
abstract
verbatim: good for adding small snippets of program code (cant use tabs, only spaces)
figure
table
tabular{options}:
	Options:
		l|r:
enumerate: list of items in an enumerated list
	\item: starts each item
itemize: list of items in bulleted list
	\item: starts each item
align: align all the lines within (mathmode)
equation: (mathmode)




Bibliography:
\bibliographystyle{style}: different styles for references
\bibliography{fileName}: pulls reference data from "fileName"












